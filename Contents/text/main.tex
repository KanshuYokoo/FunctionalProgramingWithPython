\documentclass[11pt,a4paper,twoside]{book}

% --- PACKAGES ---
\usepackage[utf8]{inputenc}
\usepackage[T1]{fontenc}
\usepackage{lmodern}
\usepackage[margin=1in]{geometry}
\usepackage{amsmath,amsfonts,amssymb}
\usepackage{graphicx}
\usepackage{hyperref}
\usepackage{xcolor}
\usepackage{listings}
\usepackage{titlesec}
\usepackage{microtype}

% --- SETUP LISTINGS FOR CODE ---
\definecolor{codegreen}{rgb}{0,0.6,0}
\definecolor{codegray}{rgb}{0.5,0.5,0.5}
\definecolor{codepurple}{rgb}{0.58,0,0.82}
\definecolor{backcolour}{rgb}{0.95,0.95,0.92}

\lstdefinestyle{mystyle}{
    backgroundcolor=\color{backcolour},   
    commentstyle=\color{codegreen},
    keywordstyle=\color{magenta},
    numberstyle=\tiny\color{codegray},
    stringstyle=\color{codepurple},
    basicstyle=\ttfamily\footnotesize,
    breakatwhitespace=false,         
    breaklines=true,                 
    captionpos=b,                    
    keepspaces=true,                 
    numbers=left,                    
    numbersep=5pt,                  
    showspaces=false,                
    showstringspaces=false,
    showtabs=false,                  
    tabsize=2
}
\lstset{style=mystyle}

% --- METADATA ---
\title{
    \vspace{2cm}
    \Huge \textbf{Functional Foundations} \\
    \vspace{0.5cm}
    \Large A Mathematical \& Practical Guide to Functional Programming and Category Theory
}
\author{
    \Large Author Name \\ % User can replace this
    \vspace{1cm}
}
\date{\today}

\begin{document}

% --- FRONTMATTER ---
\frontmatter

\maketitle

\chapter*{Preface}
This textbook bridges the gap between formal mathematics (specifically Category Theory) and practical software engineering using Python and Haskell. It is designed for students with a high school mathematical background.

\tableofcontents

% --- MAINMATTER ---
\mainmatter

% Include Chapter 1
\include{chapter1_introduction/chapter1_main}

% Include Chapter 2
\chapter{The Calculus of Functions}

\section{Introduction to Lambda Calculus}

In the 1930s, mathematician Alonzo Church invented a formal system called \textbf{Lambda Calculus} ($\lambda$-calculus) to investigate the foundations of mathematics. While Alan Turing was inventing the Turing Machine (the theoretical basis for imperative programming languages like C, Java, and Python's procedural aspects), Church was inventing Lambda Calculus, which became the theoretical foundation for all functional programming languages.

Lambda Calculus is astonishingly simple. It consists of only three things:
\begin{enumerate}
    \item \textbf{Variables:} Symbols like $x, y, z$.
    \item \textbf{Abstractions (Function Definitions):} Creating a function, denoted by the Greek letter Lambda ($\lambda$).
    \item \textbf{Applications (Function Calls):} Applying a function to an argument.
\end{enumerate}

Let us look at a simple mathematical function: $f(x) = x + 1$.
In Lambda Calculus, we drop the name ``$f$'' entirely and define just the mapping. We write it using a lambda abstraction:
\[ \lambda x . x + 1 \]

This reads as: ``A function that takes a variable $x$ (the part before the dot) and returns $x + 1$ (the body after the dot).''

\subsection{Anonymous Functions in Code}

Because these functions don't have a name like ``$f$'', they are called \textbf{Anonymous Functions}. Modern programming languages have adopted this concept directly.

Here is how you write anonymous functions in Python:

\lstinputlisting[language=Python, caption=Python: Lambda (Anonymous) Functions]{../code/python/chapter2/lambdas.py}

And here is the exact same concept in Haskell, which uses the backslash \texttt{$\backslash$} as a typographical substitute for the Greek letter $\lambda$:

\lstinputlisting[language=Haskell, caption=Haskell: Anonymous Functions]{../code/haskel/chapter2/lambdas.hs}

\section{Bound vs. Free Variables}

To understand how Lambda Calculus evaluates, we must distinguish between bound and free variables.

Look at this expression:
\[ \lambda x . x + y \]

\begin{itemize}
    \item The variable $x$ is \textbf{bound}. It is declared in the ``head'' ($\lambda x$) of the function. Whoever calls the function will provide the value for $x$.
    \item The variable $y$ is \textbf{free}. It is not defined in the head. For this function to make sense, $y$ must be defined somewhere else in the surrounding environment.
\end{itemize}

A lambda expression where every variable is bound is called a \textbf{Combinator}. The simplest combinator is the Identity Function ($\lambda x . x$), which just returns whatever is passed to it.

\section{The Rules of Computation}

Turing Machines compute by moving tape and changing states. Lambda Calculus computes using just two simple text-replacement rules: Alpha-conversion and Beta-reduction.

\subsection{Alpha-Conversion ($\alpha$-conversion)}

Alpha-conversion is the rule of \textbf{renaming}. It states that the specific name of a bound variable does not matter.

\[ \lambda x . x \quad \text{is exactly the same as} \quad \lambda y . y \]

In Python, \verb|lambda x: x| and \verb|lambda y: y| are identical functions. Changing the variable name does not change what the function \textit{does}. This allows us to avoid naming collisions when applying functions to other functions.

\subsection{Beta-Reduction ($\beta$-reduction)}

Beta-reduction is the rule of \textbf{application}. It is the actual process of ``running'' or ``calling'' the function. It works by substituting the input argument into the body of the function wherever the bound variable appears.

Consider applying our $+1$ function to the number $5$:
\[ (\lambda x . x + 1)\ 5 \]

To perform a Beta-reduction, we take the input ($5$), replace every instance of the bound variable ($x$) in the body with it, and drop the lambda head:
\[ (\lambda x . x + 1)\ 5 \xrightarrow{\beta} 5 + 1 \rightarrow 6 \]

If you string multiple lambda applications together, Beta-reduction is how you ``execute'' the program, step-by-step, until you arrive at the final answer.

\section{Why This Matters}

In functional programming, whether you are using Python's \verb|map()| or building complex Haskell architectures, your program is fundamentally a massive expression of nested functions. The compiler or interpreter executes your program by performing Beta-reductions until no more functions can be applied.

Understanding Lambda Calculus gives you the mental model to see through the syntax of a language and understand the pure dataflow underneath.


% Future chapters will be included here like so:
% \chapter{Introduction to Mathematical Functions}

\section{What is a Function?}

In mathematics, a function is a fundamental concept that describes a reliable relationship between two sets of data. When we say a function "$f$" takes an input "$x$" and produces an output "$y$", we write this relationship as:

\[ f(x) = y \]

You can think of a mathematical function as a well-behaved machine. If you drop a specific wooden block (the input) into the top of the machine, it will *always* paint it red (the output) and drop it out the bottom. It never paints it blue one day and red the next; the result is strictly determined by the input.

\subsection{Domains and Codomains}

To properly use our "machine," we need to know exactly what kind of inputs it accepts and what kind of outputs it promises to produce. In formal mathematics, we define these boundaries using **Sets**.

\begin{itemize}
    \item \textbf{Domain:} The set of all possible \textit{valid inputs}. If our machine only accepts square wooden blocks, the Domain is the set of all square wooden blocks.
    \item \textbf{Codomain:} The set of all \textit{possible outputs} the function could mathematically produce. If our machine paints blocks red, the Codomain is the set of all red objects.
\end{itemize}

When defining a function formally, mathematicians write its signature like this:

\[ f : A \rightarrow B \]

This is read as "$f$ is a function from set $A$ to set $B$". Set $A$ is the domain, and set $B$ is the codomain. Notice how this resembles an arrow pointing from the source to the destination.

\section{From Mathematics to Code}

In standard imperative programming (like writing typical Python scripts), a "function/method" is often just a sequence of instructions grouped together. It might take an input, look at a global variable, print something to the screen, change a database, and then return a value. 

That is \textbf{not} a true mathematical function. A true mathematical function (often called a "pure function" in programming) has two strict rules:
\begin{enumerate}
    \item It must always produce the exact same output for the same input.
    \item It must not cause any observable side effects (no printing, no modifying global variables).
\end{enumerate}

Let us look at how mathematics translates to code using Python and Haskell.

\subsection{A Simple Mapping}

Consider a mathematical function that squares an integer:
\[ f : \mathbb{Z} \rightarrow \mathbb{Z} \]
\[ f(x) = x^2 \]

Here is how we represent that pure mathematical function in Python:

\lstinputlisting[language=Python, caption=Python: A pure squaring function]{code/python/chapter1/functions.py}

Notice the use of type hints (`int`). In functional programming, types are the equivalent of mathematical sets. The type hint `int -> int` is the exact programmatic equivalent of the mathematical signature $\mathbb{Z} \rightarrow \mathbb{Z}$.

Now, let us look at Haskell, a pure functional language. In Haskell, you are \textit{forced} to write pure mathematical functions. 

\lstinputlisting[language=Haskell, caption=Haskell: A pure squaring function]{code/haskel/chapter1/functions.hs}

In Haskell, the signature `square :: Int -> Int` is front and center. It explicitly declares that `square` is a mapping from the set of Integers to the set of Integers.

\section{An Early Glimpse at Category Theory}

You have now seen three concepts:
\begin{itemize}
    \item Sets of data (like Integers or Strings).
    \item Functions that map data from one set to another.
    \item The idea that these mappings behave like arrows ($A \rightarrow B$).
\end{itemize}

Without realizing it, you have just stepped into \textbf{Category Theory}. 

Category Theory is an abstract branch of mathematics that studies structures and systems of structures. Instead of looking at the specific data \textit{inside} a set (like the number $4$ inside the set of integers), Category Theory zooms out and looks at how entire sets relate to each other through functions.

A **Category** is simply a collection of two things:
\begin{enumerate}
    \item \textbf{Objects:} These are the "things" in your category. In programming, you can think of Objects as your \textbf{Types} (like `int`, `str`, or `List`). In mathematics, these are often Sets.
    \item \textbf{Morphisms (Arrows):} These are the connections between the Objects. In programming, these are your \textbf{Functions}. If you have a function that takes an `int` and returns a `str`, that function is a morphism pointing from the `int` object to the `str` object.
\end{enumerate}

When you write a pure function in Python or Haskell, you are defining a morphism between two objects in the "Category of Types" (often called \textbf{Hask} in the Haskell world). 

As we progress through this book, we will rely heavily on this simple visual metaphor. By treating types as dots (objects) and functions as arrows (morphisms) connecting those dots, complex programming concepts will become elegantly simple.


% --- BACKMATTER ---
\backmatter

\end{document}
